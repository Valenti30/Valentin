

 {\itshape Proyecto creado por Valentín Gracia Sastre} 



\section*{Proyecto e\+B\+O\+A\+RD}

El proyecto consiste en un programa cliente y uno servidor que se comunican entre ellos mediante websocket y mensajes J\+S\+ON.

Se almacenan datos de jugadores y usuarios en una base de datos y se pueden crear usuarios y jugadores.

\subsection*{En qué consiste e\+B\+O\+A\+RD}

Es una aplicación diseñada para I\+OS y Android. Está enfocada para qué clubes adquieran este servicio para hacer que el entrenador pueda realizar su trabajo de forma más sencilla. En la aplicación podrá crear sus jugadores, hacer sus quintetos inicial, llevar las estadísitcas de cada uno de los jugadores...

\subsection*{Pre-\/requisitos}

Para el proyecto se necesita la libreria I\+X\+Web\+Socket para poder crear el servidor Websocket. Para poder descargarlo entra en\+: \href{https://github.com/machinezone/IXWebSocket}{\tt I\+X\+Web\+Socket}.

También se necesita una libreria para poder manejar J\+S\+ON. Para poder descargarlo entra en\+: \href{https://github.com/nlohmann/json}{\tt nlohmann/json}.

\subsection*{Estado de la aplicación}

En estos momentos la aplicación se encuentra en desarrollo, las funciones que se pueden realizar son\+: Iniciar sesión, registrarse y crear jugadores.

\subsection*{Conceptos que faltan por implementar en la aplicación}

{\bfseries Diferentes funcionalidades como}
\begin{DoxyItemize}
\item Crear personal del staff\+: son los ayudantes del entrenador
\item Crear un marcador en tiempo real, para que el entrenador pueda utilizarlo en los entrenamientos y partidos
\item Ver los equipos a los cuáles se va a enfrentar el equipo. Cuando entres dentro de cada equipo tendrás diferentes posibilidades\+:
\begin{DoxyEnumerate}
\item Saldrá el mejor jugador de ese equipo en función de los puntos por partido, asistencias por partido, rebotes por partido... que lleve a lo largo de la temporada
\item Se verá quién es el entrenador del otro equipo para saber cual es su forma de jugar los partidos, que tácticas utiliza...
\end{DoxyEnumerate}
\item Se podrán crear equipos visitantes por si nuestro equipo algún día hace algún amistoso y no tiene registrado ese equipo.
\item se podrán guardar las estadísitcas de cada uno de los jugadores de nuestro equipo y cuando el entrenador lo desee podrá sacar un análisis o un informe con unas gráficas para verlo todo más visual.
\end{DoxyItemize}

\subsection*{Usabilidad de la aplicación}


\begin{DoxyItemize}
\item Cuando entras en la aplicación pide que te loguees, en caso de que no tengas ningún usuario, deberás pulsar en el botón de registrase.
\item Si pulsas el botón se te abrirá una ventana para poder registrarte, una vez finalizado te pedirá que te loguees. Cuando lo hayas hecho ya podrás empezar a disfrutar del programa.
\end{DoxyItemize}

\subsection*{Tecnología empleada}

{\bfseries \mbox{\hyperlink{classServidor}{Servidor}}}
\begin{DoxyItemize}
\item C++ El lenguaje de programación elegido para el servidor.
\item Qt El framework de c++ usado en el proyecto.
\end{DoxyItemize}

{\bfseries Cliente}
\begin{DoxyItemize}
\item H\+T\+ML Lenguaje de marca para crear la página del cliente.
\item C\+SS Para darle diseño a la página.
\item JS Para interactuar con el servidor y enviar peticiones.
\end{DoxyItemize}

{\bfseries Base de datos}
\begin{DoxyItemize}
\item Nombre de la B\+B\+DD\+: Aplicacion\+Baloncesto
\item Php\+Pg\+Admin Es una página web que administra bases de datos Postgress\+S\+QL
\item Usuarios disponibles para hacer el login\+:
\item 1\+: usuario -\/$>$ valen, pass -\/$>$ 1234
\item 2\+: usuario -\/$>$ javi, pass -\/$>$ javi
\end{DoxyItemize}

\subsection*{Autores}


\begin{DoxyItemize}
\item Valentín Gracia Sastre (Desarrollo)
\end{DoxyItemize}

\subsection*{Licencia}

{\bfseries The M\+IT License (M\+IT)}

{\bfseries Copyright (c) 2020-\/present Valentín Gracia}

Por la presente, se otorga permiso, de forma gratuita, a cualquier persona que obtenga una copia de este software y los archivos de documentación asociados (el \char`\"{}\+Software\char`\"{}), para negociar el Software sin restricciones, incluidos, entre otros, los derechos de uso, copia, modificación, fusión, publicar, distribuir, sublicenciar y / o vender copias del Software, y permitir que las personas a quienes se les proporcione el Software lo hagan.

\subsection*{Gestión de errores}


\begin{DoxyItemize}
\item Problemas al conectar la base de datos con el servidor, conectar cliente y servidor.
\item Problemas para hacer las traducciones, me faltaba el paso de distribuir para que funcionaran.
\item Problemas con el lenguaje java\+Script, ya que no tenia conocimientos previos.
\item Problemas al recoger los datos de un mensaje J\+S\+ON.
\item Problemas para hacer el certificado, la clave pública ascii. 
\end{DoxyItemize}